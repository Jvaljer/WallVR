
\subsection*{Effects of Display Size and Navigation Type on a Classification Task}
    \subparagraph{by : C.Liu, O.Chapuis, M.Beaudouin-Lafon, E.Lecolinet, W.Mackay}
    \cite{liu2014effects}

    \begin{multicols}{2}
        \paragraph{ \textit{Quick Summary :} 
                \newline }
        \indent \indent As wall-sized displays are now becoming one of the most used technologies when it comes to data manipulations, knowing the benefits and drawbacks of using LHRD seems as interesting as important. And so this article does 
        it by comparing these to their desktop monitors  equivalents, for a task that involves explicit data manipulation. The purpose of the article is to conduct a controlled experiment to compare the performance of physical navigation in 
        front of a wall-size display with virtual navigation using pan-and-zoom on a desktop monitor for this task. The authors aim to understand the interaction effect between display type and task difficulty and to develop guidelines 
        for designing displays for interactive tasks that involve data manipulation. Overall, the article seeks to contribute to a deeper understanding of the advantages and drawbacks of wall-size displays compared to desktop monitors and to 
        provide insights for the design of displays for complex tasks.

        \paragraph{ \textit{Experiment Idea :} 
                \newline }
        \indent \indent The conducted experiment was an abstract classification task where users had to divide \& sort items into classes depending on their properties. Searchers used a middle ground approach where there were more containers than classes, 
        and then users would have to place those items into containers without letting the container overflow. The task required users to determine when two items were in the same class, which was represented by a different letter. Information density was 
        operationalized by adjusting font size. The experimenters controlled the complexity of the task through several parameters, including the number of items, classes, containers, and label font size. This approach provides a rich yet easy-to-control 
        design space for experimental tasks based on the abstract task. The experimenters compared the benefits and drawbacks of wall-sized displays and desktop displays in this classification task.

        \paragraph{ \textit{First Experiment :} 
                \newline }
        \indent \indent This first experiment compared user's performance, subjective experiences, and preferences when distinctly using a desktop computer and a large wall display for a pick-and-place task. The experiment used three different label sizes 
        and two levels of task difficulty. The results show that participants' performances were significantly better on the wall display than on the desktop, especially for the harder task and the medium label size. The wall display also resulted in lower 
        subjective mental load and frustration. Almost all participants preferred the wall display, except for the large label size. The study found a strong interaction effect between the display type and the label size, where the desktop with large labels 
        was fast but exploring small and medium labels was painful. Participants also reported different senses of engagement between the desktop and the wall display, where the desktop gave a sense of control while the wall display gave a sense of being part 
        of the interaction.

        \paragraph{ \textit{Second Experiment :} 
                \newline }
        \indent \indent This second experiment aims to test whether virtual navigation techniques on a desktop interface can beat physical navigation on a wall-size display for difficult classification tasks. In this one, researchers compared three different desktop 
        techniques for difficult classification tasks: a baseline pan-and-zoom (PZ) technique, an overview+detail (PZ+OV) technique, and a focus+context (Fisheye) technique. They recruited 12 volunteers aged 22 to 38 with normal or corrected vision and used a within-subjects design. 
        The results showed no significant difference in task completion time between the three techniques, and none of the techniques were as effective as physical navigation on a wall-size display for this task. Nine participants preferred the Fisheye technique, while three preferred PZ+OV.*
        But still, it  showed that physical navigation on a wall-size display outperformed virtual navigation using desktop techniques for difficult classification tasks. Despite the fact that the focus+context and overview+detail desktop techniques performed similarly to the pan-and-zoom 
        technique in terms of task completion time, none of them came close to the performance of physical navigation on the wall-size display. Additionally, while the fisheye lens technique was preferred by some participants, others found it difficult to control and focus on labels despite its 
        high magnification factor. Overall, the experiment confirmed the superiority of physical navigation on a wall-size display for complex data manipulation tasks compared to desktop techniques.

        \paragraph{ \textit{Discussion :}
                \newline }
        \indent \indent In the \textit{CONCLUSION AND FUTURE WORK} part, the statement is that a wall-size display can be more efficient than a desktop display for difficult data classification tasks that involve data manipulation, especially when there is a high information density. 
        Moreover, thanks to the experiment that compared physical navigation in front of a wall-size display with virtual navigation on a desktop display. Authors found that the desktop display was more efficient for easy tasks, but the wall-size display was significantly more efficient 
        (up to 35\%) for difficult tasks. They suggest that this is just the first step in understanding the benefits of wall-size displays and that future research should investigate collaborative work and new techniques for improving user performance and reducing cognitive load in 
        both wall and desktop environments.

        \paragraph{ \textit{Conclusion :}
                \newline }
        \indent \indent My personal conclusion would be that the choice of display size and navigation technique can significantly impact user performance for certain types of tasks. This may have implications for the design of user interfaces and visualizations, especially for complex 
        data manipulation tasks. The study highlights the need for further research to understand the interaction environment provided by wall-size displays, particularly in collaborative work settings. Additionally, the study suggests that a deeper understanding of spatial memory and 
        the respective advantages of physical and virtual navigation could inform the design of new techniques for both wall and desktop environments to improve user performance and reduce cognitive load.
    \end{multicols}